\documentclass[11pt]{article}
\usepackage{graphicx} % more modern
%\usepackage{times}
\usepackage{helvet}
\usepackage{courier}
\usepackage{epsf}
\usepackage{amsmath,amssymb,amsfonts,amsthm,verbatim}
\usepackage{subfigure}
\usepackage{amsfonts}
\usepackage{amsmath}
\usepackage{latexsym}
\usepackage{algpseudocode}
\usepackage{algorithm}
%\usepackage{algorithmic}
\usepackage{multirow}
\usepackage{xcolor}

\def\A{{\bf A}}
\def\a{{\bf a}}
\def\B{{\bf B}}
\def\b{{\bf b}}
\def\C{{\bf C}}
\def\c{{\bf c}}
\def\D{{\bf D}}
\def\d{{\bf d}}
\def\E{{\bf E}}
\def\e{{\bf e}}
\def\F{{\bf F}}
\def\f{{\bf f}}
\def\G{{\bf G}}
\def\g{{\bf g}}
\def\k{{\bf k}}
\def\K{{\bf K}}
\def\H{{\bf H}}
\def\I{{\bf I}}
\def\L{{\bf L}}
\def\M{{\bf M}}
\def\m{{\bf m}}
\def\n{{\bf n}}
\def\N{{\bf N}}
\def\BP{{\bf P}}
\def\R{{\bf R}}
\def\BS{{\bf S}}
\def\s{{\bf s}}
\def\t{{\bf t}}
\def\T{{\bf T}}
\def\U{{\bf U}}
\def\u{{\bf u}}
\def\V{{\bf V}}
\def\v{{\bf v}}
\def\W{{\bf W}}
\def\w{{\bf w}}
\def\X{{\bf X}}
\def\Y{{\bf Y}}
\def\Q{{\bf Q}}
\def\x{{\bf x}}
\def\y{{\bf y}}
\def\Z{{\bf Z}}
\def\z{{\bf z}}
\def\0{{\bf 0}}
\def\1{{\bf 1}}


\def\hx{\hat{\bf x}}
\def\tx{\tilde{\bf x}}
\def\ty{\tilde{\bf y}}
\def\tz{\tilde{\bf z}}
\def\hd{\hat{d}}
\def\HD{\hat{\bf D}}

\def\MA{{\mathcal A}}
\def\MR{{\mathcal R}}
\def\MF{{\mathcal F}}
\def\MG{{\mathcal G}}
\def\MI{{\mathcal I}}
\def\MN{{\mathcal N}}
\def\MO{{\mathcal O}}
\def\MT{{\mathcal T}}
\def\MX{{\mathcal X}}
\def\SW{{\mathcal {SW}}}
\def\MW{{\mathcal W}}
\def\MY{{\mathcal Y}}
\def\BR{{\mathbb R}}
\def\BP{{\mathbb P}}

\def\bet{\mbox{\boldmath$\beta$\unboldmath}}
\def\epsi{\mbox{\boldmath$\epsilon$}}

\def\etal{{\em et al.\/}\,}
\def\tr{\mathrm{tr}}
\def\rk{\mathrm{rk}}
\def\diag{\mathrm{diag}}
\def\dg{\mathrm{dg}}
\def\argmax{\mathop{\rm argmax}}
\def\argmin{\mathop{\rm argmin}}
\def\vecd{\mathrm{vec}}

\def\ph{\mbox{\boldmath$\phi$\unboldmath}}
\def\vp{\mbox{\boldmath$\varphi$\unboldmath}}
\def\pii{\mbox{\boldmath$\pi$\unboldmath}}
\def\Ph{\mbox{\boldmath$\Phi$\unboldmath}}
\def\pss{\mbox{\boldmath$\psi$\unboldmath}}
\def\Ps{\mbox{\boldmath$\Psi$\unboldmath}}
\def\muu{\mbox{\boldmath$\mu$\unboldmath}}
\def\Si{\mbox{\boldmath$\Sigma$\unboldmath}}
\def\lam{\mbox{\boldmath$\lambda$\unboldmath}}
\def\Lam{\mbox{\boldmath$\Lambda$\unboldmath}}
\def\Gam{\mbox{\boldmath$\Gamma$\unboldmath}}
\def\Oma{\mbox{\boldmath$\Omega$\unboldmath}}
\def\De{\mbox{\boldmath$\Delta$\unboldmath}}
\def\de{\mbox{\boldmath$\delta$\unboldmath}}
\def\Tha{\mbox{\boldmath$\Theta$\unboldmath}}
\def\tha{\mbox{\boldmath$\theta$\unboldmath}}

\newtheorem{theorem}{Theorem}[section]
\newtheorem{lemma}{Lemma}[section]
\newtheorem{definition}{Definition}[section]
\newtheorem{proposition}{Proposition}[section]
\newtheorem{corollary}{Corollary}[section]
\newtheorem{example}{Example}[section]


\def\probin{\mbox{\rotatebox[origin=c]{90}{$\vDash$}}}

\def\calA{{\cal A}}



%this is a comment

%use this as a template only... you may not need the subsections,
%or lists however they are placed in the document to show you how
%do it if needed.


%THINGS TO REMEMBER
%to compile a latex document - latex filename.tex
%to view the document        - xdvi filename.dvi
%to create a ps document     - dvips filename.dvi
%to create a pdf document    - dvipdf filename.dvi
%{\bf TEXT}                  - bold font TEXT
%{\it TEXT}                  - italic TEXT
%$ ... $                     - places ... in math mode on same line
%$$ ... $$                   - places ... in math mode on new line
%more info at www.cs.wm.edu/~mliskov/cs423_fall04/tex.html


\setlength{\oddsidemargin}{.25in}
\setlength{\evensidemargin}{.25in}
\setlength{\textwidth}{6in}
\setlength{\topmargin}{-0.4in}
\setlength{\textheight}{8.5in}


%%%%%%%%%%%%%%%%%%%%%%%%%%%%%%%%%%%%%%%%%%%%%%%%%%%%%%%%%%%%%%%%%%%%%%%%%%%%%%%%%%%
\newcommand{\notes}[5]{
	\renewcommand{\thepage}{#1 - \arabic{page}}
	\noindent
	\begin{center}
	\framebox{
		\vbox{
		\hbox to 5.78in { { \bf Statistical Machine Learning}
		\hfill #2}
		\vspace{4mm}
		\hbox to 5.78in { {\Large \hfill #5 \hfill} }
		\vspace{2mm}
		\hbox to 5.78in { {\it #3 \hfill #4} }
		}
	}
	\end{center}
	\vspace*{4mm}
}

\newcommand{\ho}[5]{\notes{#1}{Random Variables}{Professor: Zhihua Zhang}{}{Lecture Notes #1: Random Variables}}
%%%%%%%%%%%%%%%%%%%%%%%%%%%%%%%%%%%%%%%%%%%%%%%%%%%%%%%%%%%%%%%%%%%%%%%%%%%%%%%%%%

%begins a LaTeX document
\begin{document}

\setcounter{section}{1}
\ho{2}{2011.02.21}{Moses Liskov}{Name}{Lecture title}

\section{Random Variables}
\begin{definition}
A random variable $X$ is a measure map $X: \Omega \to  \BR$ that assigns a real number $X(\omega)$ to each out come  $\Omega$ and "measurable" means that for every $X$, $\{\omega: X(\omega) \leq x\} \in \MA$.
\end{definition}

\begin{example}
Flip a coin ten times. Let $X(\omega)$ be a number of heads in the sequence $\omega$. If $w=HHHTTTHHTT$, $X(\omega)$=5. 
\end{example}

\begin{example}
Let $\Omega = \{(x, y) | x^2 + y^2 \leq 1\}$. Consider drawing a point at random from $\Omega$. $\omega=(x,y) \in \Omega$, $X(\omega)=x, X(\omega)=y, X(\omega)=x+y$ are possible random variables.
\end{example}

\begin{definition}
Let $A \subset \BR$, $X^{-1} = \{\omega \in \Omega: X(\omega) \in A \} \in \MA$. 
$P(X \in A) \triangleq P(X^{-1}(A)) = P(\{\omega \in \Omega | X(\omega) \in A \})$.
$P(X = x) = P(X^{-1}(x)) = P(\{\omega \in \Omega | X(\omega) = x\})$
\end{definition}

Note for simplicity, we will use $\{X>0\}$ to denote $\{\omega \in \Omega : X(\omega)>0\}$, $P(X>0)$ to denote $P(\{X>0\})$.

\begin{example}
Flip a coin twice and let $X$ be the number of heads.

\begin{tabular}{|c|c|c|}
  \hline
  % after \\: \hline or \cline{col1-col2} \cline{col3-col4} ...
  $\omega$ & $P(\{\omega\})$ & $X(\omega)$ \\
  TT & 1/4 & 0 \\
  TH & 1/4 & 1 \\
  HT & 1/4 & 1 \\
  HH & 1/4 & 2 \\
  \hline
\end{tabular}

\begin{tabular}{|c|c|}
    \hline
    X & P(X) \\
    0 & 1/4 \\
    1 & 1/2 \\
    2 & 1/4 \\
    \hline
\end{tabular}
\end{example}

\subsection{Distribution Function}
Cumulative distribution function (or distribution function). CDF is the function $F_X : \BR \to [0, 1]$.

$F_X(x) = P(X \leq x)$.

\begin{example}
From example 1.3, we can get

$F_X(x) = \left\{\begin{array}{cc}
             0 & x < 0 \\
             1/4 & 0 \leq x < 1 \\
             3/4 & 1 \leq x < 2 \\
             1 & x \geq 2
           \end{array}
           \right.
$
\end{example}

\begin{theorem}
Let $X$ have CDF $F$, $Y$ have CDF $G$.
If $F(x) = G(x)$ for all $x$, then $P(X \in A) = P(Y \in A)$ for all measurable $A$.
\end{theorem}

\begin{theorem}
A function $F$ mapping $\BR \to [0, 1]$ is a CDF for probability iff
\begin{enumerate}
\item $F$ is non-deceasing, $x_1 < x_2 \implies F(x_1) \leq F(x_2)$.
\item $F$ is normalized, i.e. $\lim\limits_{x \to -\infty} F(x) = 0$, $\lim\limits_{x \to +\infty}F(x) = 1$.
\item $F$ is right-continuous. $F(x) = F(x^+)$, where $F(x^+) = \lim\limits_{y \to x, y > x} F(y)$.
\end{enumerate}
\end{theorem}

Now we will get the proof of right-continuous.

{\bf Proof: }
Let $y_1 > y_2 > \cdots$, and $\lim_{n \to +\infty} y_n = x$.

Then $F(y_1) = P(Y \leq y_1)$, $F(y_2) = P(Y \leq y_2)$, \dots

Let $A_i = (-\infty, y_i]$ and $A = (-\infty, x]$.

Note that $A = \cap_{i=1}^\infty A_i$ and $A_1 \supset A_2 \supset \cdots$

$\lim_{i \to \infty} P(A_i) = P(\cap_{i=1}^\infty A_i)$.

$F(x) = P(A) = P(\cap_{i=1}^\infty A_i) = \lim_{i\to\infty} P(A_i) = \lim_{i \to \infty} F(y_i) = F(x^+)$.

\end{document} 